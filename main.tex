\documentclass[12pt,a4paper]{article}

\usepackage[utf8]{inputenc}
\usepackage[T1]{fontenc}
\usepackage{lmodern}
\usepackage{geometry}
\usepackage{fancyhdr}
\usepackage[hidelinks]{hyperref}
\setlength{\headheight}{14.49998pt}
\usepackage{graphicx}
\usepackage[sorting=none]{biblatex}
\addbibresource{references.bib}
\usepackage{lastpage} 
\usepackage{svg}
\usepackage{titlesec}
\usepackage[french]{babel}
\usepackage{lipsum} % Pour le texte d'exemple
\graphicspath{{img/}}
\geometry{left=2.5cm,right=2.5cm,top=3cm,bottom=3cm}

\newcommand{\Title}{ACV du parc informatique de l’ENSEIRB-MATMECA} 
\newcommand{\Course}{Développement durable et responsabilité sociétale
\\ Département Informatique
\\ S6 - Année 2024/2025} 
\newcommand{\Author}{Mélissa Colin - Noé Offredo}
\newcommand{\Date}{\today}
\pagestyle{fancy}
\lhead{\textbf{\Author}} 
\rhead{1A - Informatique}
\cfoot{\thepage/\pageref{LastPage}}



% Configuration des sections pour qu'elles soient centrées et commencent sur une nouvelle page
\newcommand{\sectionbreak}{\clearpage} % Force une page avant chaque section
\titleformat{\section} % Format des sections
    {\centering\Large\bfseries} % Centré, grand, gras
    {\thesection} % Numéro de section
    {1em} % Espacement
    {} % Style du titre

\begin{document}

\begin{titlepage}
    \centering
    \vspace*{3cm}
    {\Huge \textbf{\Title}}\\[1.5cm]
    {\Large \textit{\Course}}\\[2cm]
    \textbf{\Author}\\[1cm]
    \Date\\
    \vfill
    \includegraphics[width=0.3\textwidth]{logo.png}
    \vspace*{1cm}
\end{titlepage}

\tableofcontents
\newpage

\section{Introduction}
\subsection{Contexte et motivation}
L’objectif de ce TP est de comparer les impacts environnementaux de 2 solutions numériques
pour le parc informatique de l’école à l’aide d’openLCA et d’ecoinvent.
Ce TP a une durée de 5 h environ. Un compte-rendu de 4 pages (2 feuilles
recto/verso) est attendu. Il faudra déposer ce rapport sur Moodle, en PDF.
Il y a des questions bonus. Vous devez en choisir au moins trois pour votre CR. Ces questions
sont relativement larges et doivent demander une étude approfondie.
L’objectif de ce projet est de comparer 2 solutions pour renouveler le parc informatique de
l’ENSEIRB-MATMECA :
• 600 ordinateurs Dell de type tour T1700/3620,
• 600 Raspberry Pi 4 Rev B (processeur ARM v8 Cortex-A72) connectés à 6 serveurs Dell
Precision 7920T équipés de deux processeurs Intel Xeon Platinum 8168.
Nous allons intégrer dans cette comparaison seulement ce qui est différent (fabrication et uti-
lisation des deux solutions) et nous n’allons donc pas considérer l’écran ou les périphériques
d’entrée (clavier, souris, etc.). La fin de vie ne sera pas prise en compte. C’est un champ de
recherche et de développement industriel en plein essor et il est difficile pour le moment de
facilement la modéliser. C’est un des premiers points d’amélioration possibles de cette étude.
L’unité fonctionnelle est :
Utilisation de 600 ordinateurs à l’ENSEIRB-MATMECA pendant 5 ans.
Ce TP se fait en 3 étapes :
I. Nous allons commencer par modéliser l’ordinateur de type tour T1700/3620. Des briques
sont disponibles (les processus [Motherboard], [RAM] et [GPU] et les providers associés
ont été créés lors d’un projet 2020/2021 par des étudiants de 2A de la filière Électronique).
II. Les processus [Raspberry Pi 4 Rev B] et [Server 48 cores] sont déjà créés. Il ne reste qu’à
créer le processus d’utilisation couplée de ces 2 technologies.
III. Enfin, une comparaison des 2 stratégies sera mise en place.
Les processus à utiliser sont donnés en annexe. N’hésitez pas à réfléchir et à
critiquer (de manière constructive !) l’ensemble des étapes de ce TP. Pensez à bien
indiquer les sources des résultats de vos recherches
\subsection{Plan}
\noindent \lipsum[1-2]

\section{Méthodologie}
\lipsum[1-2]
\subsection{Ordinateur Dell de type tour T1700/3620}
Créer le processus Computer DELL T1700/3620 dans le dossier Processes/2 : TP2 - Compu-
ters/. L’unité est Number of items. Nous en considérerons 600 quand nous ferons le processus
d’utilisation de cette stratégie. Ne pas oublier de compléter aussi la colonne Description dans la
liste des Inputs/Outputs pour ne pas vous perdre dans toutes les étapes. Rappel : le transport
des matières premières sur le site de fabrication sera intégré en ayant recours à des providers de
type market for. Nous supposons que l’ordinateur est fabriqué en Chine.
1. Voici les différents composants que vous devez intégrer à votre ordinateur (Table 1) :
(a) La carte mère - Elle accueille l’ensemble des composants internes de votre ordinateur
(processeur, mémoire. . .) et gère les différentes interfaces avec vos périphériques : prises
pour les éléments internes et ports USB pour les périphériques externes.
(b) Le processeur - Il permet de manipuler et de traiter les données qui lui sont fournies.
Sa puissance a une influence sur la vitesse d’exécution des logiciels et des opérations
effectuées.
(c) La mémoire vive - Elle stocke temporairement les données à traiter par le processeur.
Ainsi, plus il y a de mémoire de disponible, plus il est possible d’y conserver de données
temporaires (ce qui évite d’accéder au disque dur qui est plus lent). La mémoire vive est
vidée à chaque arrêt ou redémarrage de votre ordinateur.
(d) Le disque dur - Contrairement à la mémoire vive, les données stockées sur un disque dur
sont permanentes et ne sont pas effacées à l’arrêt de votre ordinateur. C’est donc sur ce
disque que votre système d’exploitation, vos logiciels et vos documents sont conservés.
(e) L’alimentation - Elle a pour rôle d’assurer la fourniture en électricité de tous les com-
posants de votre ordinateur. C’est un élément important puisque les tensions délivrées
doivent rester stables même lorsque l’alimentation est très sollicitée.
(f) La carte graphique - Elle permet de traiter et d’afficher sur un écran les données provenant
de votre ordinateur.
(g) Le lecteur / graveur CD/DVD.
[Bonus 1]. Détailler la modélisation de la carte mère. La critiquer.
2. Concernant le châssis de l’ordinateur, un processus unitaire a été créé.
[Bonus 2]. Décrire le processus du châssis en détails. Le critiquer.
3. Concernant l’assemblage, nous allons considérer l’électricité produite en Chine, l’eau potable
et son traitement (Table 2). Pour le traitement, la valeur doit être négative car c’est
considéré comme un déchet. C’est une particularité de la BDD ecoinvent. Pour les
quantités d’électricité et d’eau utilisées par l’usine d’assemblage, elles sont issues du processus
ecoinvent desktop computer et sont ramenées à une tour dont les détails sont donnés dans ce
rapport.
[Bonus 3]. Retrouver les valeurs proposées pour les quantités d’électricité et d’eau.
4. Nous allons à présent considérer l’emballage de l’ordinateur (packaging). Nous prendrons en
compte : l’emballage cartonné, les granulés de plastique pour tous les films plastiques et enfin la
transformation du plastique (polypropylène) en film (Table 2).
5. L’ordinateur est transporté du site de fabrication au site d’utilisation. Nous supposons que
ce trajet se fait en bateau (15 000 km) et en camion (1000 km en Chine et 1000 km en France
hexagonale). Vous pouvez utiliser l’unité km × kg une fois que vous aurez estimé le poids de
notre colis. Pour les processus de cette question, vous pouvez vous inspirer du TP 1 et utiliser
le processus transport, freight, sea, transoceanic ship pour la partie maritime.
6. Bravo votre processus Computer DELL T1700/3620 est terminé ! Vous pouvez donc créer le
product system associé et réaliser le calcul d’impact. Faire une comparaison avec le bilan carbone
effectué par Dell : lien.
Point d’étape : appelez votre chargé·e de TP
7. Nous allons maintenant pouvoir créer le processus correspondant à notre unité fonction-
nelle. Vous pouvez donc créer un processus Use of computer DELL T1700/3620 at ENSEIRB-
MATMECA. L’unité est la durée d’utilisation des ordinateurs. Compléter ce processus sachant
que nous souhaitons prendre en compte à la fois les 600 ordinateurs mais aussi leur consomma-
tion électrique (en phase d’usage classique, Table 3). À présent, vous pouvez créer le product
system et faire le calcul. Voici quelques questions :
a. Bilan carbone : Comment se sépare la phase de fabrication de la phase d’usage ? Et par rap-
port à l’étude de Dell ?
b. Bilan carbone : Qu’est-ce qui est le plus impactant dans l’ordinateur ?
c. Étudier d’autres facteurs d’impact comme par exemple : ionizing radiation, water consumption
et mineral ressources scarcity.
[Bonus 4]. Bilan carbone : Quel est le bilan carbone d’un étudiant ou d’une étudiante de
l’ENSEIRB-MATMECA pour la partie informatique durant ses 3 années d’études ?
\subsection{Six cent Raspberry Pi couplés à six serveurs}
On considère que 6 serveurs sont suffisants pour piloter 600 clients légers constitués de Raspberry
Pi.
8. Regarder et commenter le processus Raspberry Pi 4 Rev B. Faire de même avec le processus
Serveur 48 cœurs. Remarque : Le processus mounting correspondant ici à l’assemblage des
composants sur le circuit imprimé (placement des composants, brasure, nettoyage des cartes
assemblées à l’issue du processus, etc.). Pour l’ordinateur et le serveur, il est déjà directement
intégré au niveau des composants.
9. Créer le product system Use of Raspberry + Server at ENSEIRB-MATMECA. Ne pas oublier
de prendre en compte la consommation électrique des Raspberry Pi mais aussi des serveurs.
Répondre aux questions qui vous semblent pertinentes de la question 7


\subsection{Protocole}
\noindent \lipsum[1-2]


\section{Résulats}
\subsection{Ordinateur Dell de type tour T1700/3620}
\noindent \lipsum[1-2]

\subsection{Six cent Raspberry Pi couplés à six serveurs}
\noindent \lipsum[1-2]

\subsection{Comparaison entre les 2 stratégies}
10. Vous pouvez maintenant effectuer une comparaison entre les 2 stratégies. Quelle est votre
conclusion ? Quelles sont les limitations de l’étude ? Quels sont les avantages et les inconvénients
de chacune des stratégies ?
Point d’étape : appelez votre chargé·e de TP
11. Six serveurs sont suffisants mais en cas de panne, cela peut poser problème. Que se passe-t-il
si l’on en considère 8 ? Votre conclusion change-t-elle ?
[Bonus 5]. Que se passe-t-il si l’on fait à présent notre étude avec le mix électrique polonais ?
[Bonus 6]. Les écrans n’ont pas été modélisés. Quels sont les choix que nous aurions pu faire ?
(Rechercher par exemple des bilans carbone d’écrans).

\section{Discussion}
\noindent \lipsum[1-2]
\subsection{Limites de l'étude}
\noindent \lipsum[1-2]
\subsection{Perspectives}
\noindent \lipsum[1-2]


% Conclusion
\section{Conclusion et ouvertures}
\noindent \noindent \lipsum[1-2]
\include{glossaire}

% Références
\printbibliography
\addcontentsline{toc}{section}{Références}

\end{document}
