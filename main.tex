\documentclass[12pt,a4paper]{paper}


\newcommand{\Title}{Analyse du cycle de vie du parc informatique de l’ENSEIRB-MATMECA} 
\newcommand{\Course}{Développement durable et responsabilité sociétale
\\ Département Informatique
\\ S6 - Année 2024/2025} 
\newcommand{\Author}{Mélissa Colin - Noé Offredo}
\newcommand{\Date}{\today}
\pagestyle{fancy}
\lhead{\textbf{\Author}} 
\rhead{1A - Informatique}
\cfoot{\thepage/\pageref{LastPage}}

% \fancypagestyle{plain}{ % Apply the fancy style to plain pages like the title page
%     \lhead{\textbf{\Author}} 
%     \rhead{1A - Informatique}
%     \cfoot{\thepage/\pageref{LastPage}}
% } POUR AVOIR COMME SYR LES LIVRES


\title{\textbf{\Title}}

% plusieurs auteurs 
\author{M. Colin\textsuperscript{1}, N. Offredo\textsuperscript{1}\\[6pt]
\textsuperscript{1} ENSEIRB-MATMECA\\
}

\begin{document}

% A VOIR SI ON GARDE
\begin{titlepage}
    \centering
    \vspace*{3cm}
    {\Huge \textbf{\Title}}\\[1.5cm]
    {\Large \textit{\Course}}\\[2cm]
    \textbf{\Author}\\[1cm]
    \Date\\
    \vfill
    \includegraphics[width=0.3\textwidth]{logo.png}
    \vspace*{1cm}
\end{titlepage}
% --------------------

\maketitle

\begin{resume}
C'est mon résumé. Il doit occuper AU MAXIMUM une dizaine de lignes.
\end{resume}

\begin{motscles}
Exemple type, format, modèle.
\end{motscles}

\begin{abstract}
It's the English version of the abstract. Exactly as in French it must be short. It must speak of the same topics...  
\end{abstract}

\begin{keywords}
Example, model, template.
\end{keywords}


\section{Introduction}
\subsection{Contexte et motivation}
Alors que la transition énergétique est au coeur des préoccupations de notre société, il est indispensable de se pencher sur l'impact environnemental des technologies numériques. En effet, la fabrication et l'utilisation des équipements informatiques génèrent une empreinte carbone significative. Dans ce contexte, l'ENSEIRB-MATMECA, école d'ingénieurs spécialisée dans le numérique, s'interroge sur la durabilité de son parc informatique. Ce rapport vise à évaluer l'impact environnemental de deux solutions numériques : 600 ordinateurs Dell de type tour T1700/3620 et 600 Raspberry Pi 4 Rev B connectés à 6 serveurs Dell Precision 7920T.

\subsection{Plan}
\noindent \lipsum[1-2]

\section{Méthodologie}
Dans cetet étude, nous avons modélisé deux solutions informatiques pour l'ENSEIRB-MATMECA : 600 ordinateurs et 600 Raspberry Pi connectés à 6 serveurs. Nous avons nous concentrer sur la fabrication et l'utilisation de ces équipements, en négligeant les écrans et les périphériques d'entrée. La fin de vie des équipements ne sera pas prise en compte dans cette étude en raison de la complexité de sa modélisation. L'unité fonctionnelle retenue est l'utilisation de 600 ordinateurs pendant 5 ans à l'ENSEIRB-MATMECA.

\subsection{Outils de modélisation et base de données}
Nous avons utilisé le logiciel openLCA pour modélisé les processus de fabrication et d'utilisation des ordinateurs et des Raspberry Pi. Cet outil permet de réaliser des analyses de cycle de vie (ACV) en intégrant différentes bases de données, dont ecoinvent~\cite{ecoinvent2024}. Ecoinvent est une base de données de référence pour l'analyse du cycle de vie, fournissant des données sur les processus industriels, les flux de matières et d'énergie, ainsi que les impacts environnementaux associés. C'est pour sa richesse et sa précision que nous avons choisi cette base de données pour notre étude.


\subsection{Ordinateur Dell de type tour T1700/3620}
Dans une premiers temps, nous avons modélisé l'ordinateur Dell de type tour T1700/3620. Ces modèles, relativement équivalents, sont composés de plusieurs composants essentiels à leur fonctionnement. Pour cela nous avons réalisé un \textit{processus}\footnote{Un processus dans openLCA est un ensemble d'activités qui transforment des matières premières en produits finis.} composé de 600 \textit{Number of items} et y avons intégré les éléments suivants en fonction des indictaions fournies dans le sujet~\cite{TP2_ACV_ENSEIRB-MATMECA} pour le choix des processus de la base ecoinvent~\cite{ecoinvent2024} :
\begin{itemize}
    \item La carte mère : Motherboard
    \item Le processeur : integrated circuit, logic type
    \item La mémoire vive : RAM
    \item Le disque dur : Hard disk drive, for desktop computer
    \item L'alimentation : Power supply unit, for desktop computer
    \item La carte graphique : GPU
    \item Le lecteur/graveur CD/DVD : Disk drive, CD/DVD, ROM, for desktop computer
\end{itemize}
Nous avosn définis l'ensemble de ces composants avec une quantitée de une unitée hors mis pour la RAM, après observation directe du contenu de l'ordinateur, nous avons déterminé qu'il y avait 2 barrettes de RAM. Nous avons donc défini la RAM avec une quantité de 2 unités.\\
\textcolor{red}{A METTRE DANS LES AMELIORATION : Ces processus sont déjà intégrés dans la base de données ecoinvent. Cependant, il est important de noter que ces processus peuvent ne pas être entièrement représentatifs des modèles spécifiques de l'ordinateur Dell T1700/3620. Par conséquent, il est recommandé de vérifier les données et d'ajuster les paramètres si nécessaire pour obtenir une modélisation plus précise.}
% [Bonus 1]. Détailler la modélisation de la carte mère. La critiquer.
A cela, nous avons ajouté le chassis de l'ordinateur qui n'est pas encore modélisé dans la base de données ecoinvent. \textcolor{red}{Pour se faire nous avons .................. (un processus unitaire a été créé)}\\ \\
% [Bonus 2]. Décrire le processus du châssis en détails. Le critiquer.
Nous avons par la suite intégré le processus d'assemblage de l'ordinateur. Nous avons pour cela considéré l'électricité produite en Chine, l'eau potable et son traitement.
Le sujet~\cite{TP2_ACV_ENSEIRB-MATMECA} nous précise que l'éléctricité putilisé pour l'assemblage correspond au processus ecoinvent \textit{electricity,
medium voltage} avec une quantitée de 2,767 kWh. \textcolor{blue}{Pour vérifier cette infromation, nous avons consulté le rapport \textit{Electronic Devices - Part III}\cite{Lehmann2007} qui indique que la consommation d'énergie pour découper les plaques d'acier est de 0,277 KWh, 0,222 KWh pour le fraisage, et que l'éléctricité pour l'assemblage est de 2,222 KWh. En additionnant ces valeurs, nous obtenons une consommation totale de 2,721 KWh. Le rapport mentionne néanmoins que ses valeurs sont des moyennes et que leurs écart-type géométrique est de 95\% ce qui signifie que la consommation d'énergie peut varier considérablement en fonction des conditions de fabrication. Par conséquent, nous avons décidé de conserver la valeur de 2,767 KWh fournie dans le sujet car elle est tout tout de même proche de la valeur que nous avons calculée.}\\
Pour l'eau potable, le sujet~\cite{TP2_ACV_ENSEIRB-MATMECA} nous indique que la valeur est de 1600kg. Nous avons donc utilisé le processus ecoinvent \textit{tap water} avec cette quantitée car \textcolor{blue}{elle est confirmée par le rapport \textit{Electronic Devices - Part III}\cite{Lehmann2007} qui indique que la consommation d'eau pour l'assemblage est de 1620kg, ce qui est très proche de la valeur fournie dans le sujet.}\\
Pour le traitement de l'eau, nous avons utilisé le processus ecoinvent \textit{wastewater treatment, municipal, aerobic} avec une quantitée de -1,62m³ trouvée dans le rapport \textit{Electronic Devices - Part III}\cite{Lehmann2007}. Cette valeur est négative car elle est considérée comme un déchet et la base de données ecoinvent qui utilise la méthode \textit{Opposite Direction Approach} qui consiste à modéliser les flux de déchets en tant que flux négatifs. Cette approche permet de représenter le traitement des déchets comme une consommation négative de ressources, inversant ainsi le sens conventionnel des flux de matériaux. En conséquence, le processus de traitement de l’eau est modélisé avec une valeur négative, indiquant qu’il s’agit d’un rejet nécessitant un traitement, et non d’une ressource consommée. Cette méthodologie garantit que les impacts environnementaux du traitement des déchets sont correctement pris en compte dans l’analyse du cycle de vie~\cite{openLCATutorial2020}.
\\ \\ --------------------- \\ \\
4. Nous allons à présent considérer l’emballage de l’ordinateur (packaging). Nous prendrons en
compte : l’emballage cartonné, les granulés de plastique pour tous les films plastiques et enfin la
transformation du plastique (polypropylène) en film (Table 2).
5. L’ordinateur est transporté du site de fabrication au site d’utilisation. Nous supposons que
ce trajet se fait en bateau (15 000 km) et en camion (1000 km en Chine et 1000 km en France
hexagonale). Vous pouvez utiliser l’unité km × kg une fois que vous aurez estimé le poids de
notre colis. Pour les processus de cette question, vous pouvez vous inspirer du TP 1 et utiliser
le processus transport, freight, sea, transoceanic ship pour la partie maritime.
6. Bravo votre processus Computer DELL T1700/3620 est terminé ! Vous pouvez donc créer le
product system associé et réaliser le calcul d’impact. Faire une comparaison avec le bilan carbone
effectué par Dell : lien.
Point d’étape : appelez votre chargé·e de TP
7. Nous allons maintenant pouvoir créer le processus correspondant à notre unité fonction-
nelle. Vous pouvez donc créer un processus Use of computer DELL T1700/3620 at ENSEIRB-
MATMECA. L’unité est la durée d’utilisation des ordinateurs. Compléter ce processus sachant
que nous souhaitons prendre en compte à la fois les 600 ordinateurs mais aussi leur consomma-
tion électrique (en phase d’usage classique, Table 3). À présent, vous pouvez créer le product
system et faire le calcul. Voici quelques questions :
a. Bilan carbone : Comment se sépare la phase de fabrication de la phase d’usage ? Et par rap-
port à l’étude de Dell ?
b. Bilan carbone : Qu’est-ce qui est le plus impactant dans l’ordinateur ?
c. Étudier d’autres facteurs d’impact comme par exemple : ionizing radiation, water consumption
et mineral ressources scarcity.
[Bonus 4]. Bilan carbone : Quel est le bilan carbone d’un étudiant ou d’une étudiante de
l’ENSEIRB-MATMECA pour la partie informatique durant ses 3 années d’études ?
\subsection{Six cent Raspberry Pi couplés à six serveurs}
On considère que 6 serveurs sont suffisants pour piloter 600 clients légers constitués de Raspberry
Pi.
8. Regarder et commenter le processus Raspberry Pi 4 Rev B. Faire de même avec le processus
Serveur 48 cœurs. Remarque : Le processus mounting correspondant ici à l’assemblage des
composants sur le circuit imprimé (placement des composants, brasure, nettoyage des cartes
assemblées à l’issue du processus, etc.). Pour l’ordinateur et le serveur, il est déjà directement
intégré au niveau des composants.
9. Créer le product system Use of Raspberry + Server at ENSEIRB-MATMECA. Ne pas oublier
de prendre en compte la consommation électrique des Raspberry Pi mais aussi des serveurs.
Répondre aux questions qui vous semblent pertinentes de la question 7\cite{Dell2018}


\subsection{Protocole}
\noindent \lipsum[1-2]


\section{Résulats}
\subsection{Ordinateur Dell de type tour T1700/3620}
\noindent \lipsum[1-2]

\subsection{Six cent Raspberry Pi couplés à six serveurs}
\noindent \lipsum[1-2]

\subsection{Comparaison entre les 2 stratégies}
10. Vous pouvez maintenant effectuer une comparaison entre les 2 stratégies. Quelle est votre
conclusion ? Quelles sont les limitations de l’étude ? Quels sont les avantages et les inconvénients
de chacune des stratégies ?
Point d’étape : appelez votre chargé·e de TP
11. Six serveurs sont suffisants mais en cas de panne, cela peut poser problème. Que se passe-t-il
si l’on en considère 8 ? Votre conclusion change-t-elle ?
[Bonus 5]. Que se passe-t-il si l’on fait à présent notre étude avec le mix électrique polonais ?
[Bonus 6]. Les écrans n’ont pas été modélisés. Quels sont les choix que nous aurions pu faire ?
(Rechercher par exemple des bilans carbone d’écrans).

\section{Discussion}
\noindent \lipsum[1-2]
\subsection{Limites de l'étude}
\noindent \lipsum[1-2]
\subsection{Perspectives}
\noindent \lipsum[1-2]


% Conclusion
\section{Conclusion et ouvertures}
\noindent \noindent \lipsum[1-2]
% \include{glossaire}

% % Références
\printbibliography
\addcontentsline{toc}{section}{Références}
% \bibliography{bibliography}

\appendix

\end{document}
