\documentclass[12pt,a4paper]{paper}
\usepackage{float} % Required for the H float option
\usepackage{stfloats}


\newcommand{\Title}{Analyse du cycle de vie du parc informatique de l’ENSEIRB-MATMECA} 
\newcommand{\Course}{Développement durable et responsabilité sociétale
\\ Département Informatique
\\ S6 - Année 2024/2025} 
\newcommand{\Author}{Mélissa Colin - Noé Offredo}
\newcommand{\Date}{\today}
\pagestyle{fancy}
\lhead{\textbf{\Author}} 
\rhead{1A - Informatique}
\cfoot{\thepage/\pageref{LastPage}}

% \fancypagestyle{plain}{ % Apply the fancy style to plain pages like the title page
%     \lhead{\textbf{\Author}} 
%     \rhead{1A - Informatique}
%     \cfoot{\thepage/\pageref{LastPage}}
% } POUR AVOIR COMME SYR LES LIVRES


\title{\textbf{\Title}}

% plusieurs auteurs 
\author{M. Colin\textsuperscript{1}, N. Offredo\textsuperscript{1}\\[6pt]
\textsuperscript{1} ENSEIRB-MATMECA\\
}

\begin{document}

% A VOIR SI ON GARDE
% \begin{titlepage}
%     \centering
%     \vspace*{3cm}
%     {\Huge \textbf{\Title}}\\[1.5cm]
%     {\Large \textit{\Course}}\\[2cm]
%     \textbf{\Author}\\[1cm]
%     \Date\\
%     \vfill
%     \includegraphics[width=0.3\textwidth]{logo.png}
%     \vspace*{1cm}
% \end{titlepage}
% --------------------

\maketitle

% \begin{resume}
% C'est mon résumé. Il doit occuper AU MAXIMUM une dizaine de lignes.
% \end{resume}

% \begin{motscles}
% Exemple type, format, modèle.
% \end{motscles}

% \begin{abstract}
% It's the English version of the abstract. Exactly as in French it must be short. It must speak of the same topics...  
% \end{abstract}

% \begin{keywords}
% Example, model, template.
% \end{keywords}


\section{Introduction}
\subsection{Contexte et motivation}
Alors que la transition énergétique est au coeur des préoccupations de notre société, il est indispensable de se pencher sur l'impact environnemental des technologies numériques. En effet, la fabrication et l'utilisation des équipements informatiques génèrent une empreinte carbone significative. Dans ce contexte, l'ENSEIRB-MATMECA, école d'ingénieurs spécialisée dans le numérique, s'interroge sur la durabilité de son parc informatique. Ce rapport vise à évaluer l'impact environnemental de deux solutions numériques : 600 ordinateurs Dell de type tour T1700/3620 et 600 Raspberry Pi 4 Rev B connectés à 6 serveurs Dell Precision 7920T.

% \subsection{Plan}
% \noindent \lipsum[1-2]

\section{Méthodologie}
L’objectif de notre étude est de comparer empiriquement l'impact environnemental de deux solutions pour le parc informatique de l'ENSEIRB-MATMECA : 600 ordinateurs et 600 Raspberry Pi connectés à 6 serveurs. Nous avons nous concentrer sur la fabrication et l'utilisation de ces équipements, en négligeant les écrans et les périphériques d'entrée. La fin de vie des équipements ne sera pas prise en compte dans cette étude en raison de la complexité de sa modélisation. L'unité fonctionnelle retenue est l'utilisation de 600 ordinateurs pendant 5 ans à l'ENSEIRB-MATMECA.\\
Nous décrivons ci-dessous les outils de modélisation utilisés, ainsi que les processus de fabrication et d'utilisation des ordinateurs et des Raspberry Pi. 

\subsection{Outils de modélisation et base de données}
Nous avons utilisé le logiciel openLCA pour modélisé les processus de fabrication et d'utilisation des ordinateurs et des Raspberry Pi. Cet outil permet de réaliser des analyses de cycle de vie (ACV) en intégrant différentes bases de données, dont ecoinvent~\cite{ecoinvent2024}. Ecoinvent est une base de données de référence pour l'analyse du cycle de vie, fournissant des données sur les processus industriels, les flux de matières et d'énergie, ainsi que les impacts environnementaux associés. C'est pour sa richesse et sa précision que nous avons choisi cette base de données pour notre étude.

\subsection{Ordinateur Dell de type tour T1700/3620}
Dans une premiers temps, nous avons modélisé l'impacte environnemental des ordinateurs Dell de type tour T1700/3620 présent dans le parc informatique de l'ENSEIRB-MATMECA. Nous décrirons ci-dessous les différentes étapes de la modélisation, en commençant par la fabrication des ordinateurs, suivie de l'emballage et du transport. 
\subsubsection{Composants}
Ces modèles, relativement équivalents, contiennent plusieurs composants essentiels à leur fonctionnement. Pour cela nous avons réalisé un \textit{processus}\footnote{Un processus dans openLCA est un ensemble d'activités qui transforment des matières premières en produits finis.} modélisant un ordinateur et y avons intégré les éléments suivants en fonction des indictaions fournies dans le sujet~\cite{TP2_ACV_ENSEIRB-MATMECA} pour le choix des processus de la base ecoinvent~\cite{ecoinvent2024} :
\begin{itemize}
    \item Le processeur : integrated circuit, logic type
    \item Le disque dur : Hard disk drive, for desktop computer
    \item L'alimentation : Power supply unit, for desktop computer
    \item Le lecteur/graveur CD/DVD : Disk drive, CD/DVD, ROM, for desktop computer
\end{itemize}
A cela, nous avons ajouté le chassis de l'ordinateur, la carte mère, la RAM et la carte graphique. Ces composants ne sont pas encore modélisé dans la base de données ecoinvent. Pour se faire nous avons utilisé les providers créés lors d’un projet 2020/2021 par des étudiants de 2A de la filière Électronique.\\
Nous avons définis l'ensemble de ces composants avec une quantitée de une unitée hors mis pour la RAM, après observation visuelle du contenu de l'ordinateur, nous avons déterminé qu'un ordinateur contenait 2 barrettes de RAM. Nous l'avons donc défini avec une quantité de 2 unités.
% [Bonus 1]. Détailler la modélisation de la carte mère. La critiquer.

% [Bonus 2]. Décrire le processus du châssis en détails. Le critiquer.

\subsubsection{Fabrication}
Nous avons par la suite intégré le processus d'assemblage de l'ordinateur. Nous avons pour cela considéré l'électricité produite en Chine, l'eau potable et son traitement.
Le sujet~\cite{TP2_ACV_ENSEIRB-MATMECA} nous précise que l'éléctricité putilisé pour l'assemblage correspond au processus ecoinvent \textit{electricity,
medium voltage} avec une quantitée de 2,767 kWh. \textcolor{blue}{[Bonus 3]}~Pour vérifier cette infromation, nous avons consulté le rapport \textit{Electronic Devices - Part III}~\cite{Lehmann2007} qui indique que la consommation d'énergie pour découper les plaques d'acier est de 0,277 KWh, 0,222 KWh pour le fraisage, et que l'éléctricité pour l'assemblage est de 2,222 KWh. En additionnant ces valeurs, nous obtenons une consommation totale de 2,721 KWh. Le rapport mentionne néanmoins que ses valeurs sont des moyennes et que leurs écart-type géométrique est de 95\% ce qui signifie que la consommation d'énergie peut varier considérablement en fonction des conditions de fabrication. Par conséquent, nous avons décidé de conserver la valeur de 2,767 KWh fournie dans le sujet car elle est tout tout de même proche de la valeur que nous avons calculée.\\
Pour l'eau potable, le sujet~\cite{TP2_ACV_ENSEIRB-MATMECA} nous indique que la valeur est de 1600kg. Nous avons donc utilisé le processus ecoinvent \textit{tap water} avec cette quantitée carelle est confirmée par le rapport \textit{Electronic Devices - Part III}\cite{Lehmann2007} qui indique que la consommation d'eau pour l'assemblage est de 1620kg, ce qui est très proche de la valeur fournie dans le sujet.\\
Pour le traitement de l'eau, nous avons utilisé le processus ecoinvent \textit{wastewater, unpolluted} avec une quantitée de -1,62m³ trouvée dans le rapport \textit{Electronic Devices - Part III}\cite{Lehmann2007}. Cette valeur est négative car elle est considérée comme un déchet et la base de données ecoinvent qui utilise la méthode \textit{Opposite Direction Approach} qui consiste à modéliser les flux de déchets en tant que flux négatifs. Cette approche permet de représenter le traitement des déchets comme une consommation négative de ressources, inversant ainsi le sens conventionnel des flux de matériaux. En conséquence, le processus de traitement de l’eau est modélisé avec une valeur négative, indiquant qu’il s’agit d’un rejet nécessitant un traitement, et non d’une ressource consommée. Cette méthodologie garantit que les impacts environnementaux du traitement des déchets sont correctement pris en compte dans l’analyse du cycle de vie~\cite{openLCATutorial2020}.

\begin{figure*}[b!]
    \centering
    \includegraphics[width=\textwidth]{computer-process.png}
    \caption{Processus de fabrication de l'ordinateur}
    \label{fig:computer-process}
\end{figure*}
\subsubsection{Emballage}
Pour être transporté, l'ordinateur doit être emballé. Nous avons donc modélisé l'emballage de l'ordinateur en prenant en compte les éléments suivants :
\begin{itemize}
    \item L'emballage cartonné
    \item Les granulés de plastique pour tous les films plastiques
    \item La transformation du plastique (polypropylène) en film
\end{itemize}
Nous avons utilisé le processus ecoinvent \textit{corrugated board box} pour l'emballage cartonné avec une quantitée de 2,19kg (toujours d'après le rapport \textit{Electronic Devices - Part III}\cite{Lehmann2007}). Pour les granulés de plastique, nous avons utilisé le processus ecoinvent \textit{polypropylene,
granulate} avec une quantitée de 0,16kg. Enfin, pour la transformation du plastique en film, nous avons utilisé le processus ecoinvent \textit{polymer foaming} avec une quantitée de 0,16kg. Ces deux dernières valeurs sont également fournies dans le sujet~\cite{TP2_ACV_ENSEIRB-MATMECA}.

\subsubsection{Transport}
Nous considérons que l'ordinateur est transporté du site de fabrication au site d'utilisation en bateau (15 000 km) et en camion (1000 km en Chine et 1000 km en France). Pour cela nous avons alors utiliser le processus \textit{transport, freight, sea, transoceanic ship} pour la partie maritime et les processus \textit{transport, freight, lorry 16-32 metric ton, EURO5} avec pour provider \textit{market for $\cdots$ RER} et \textit{market for $\cdots$ RoW} respectivement pour la partie terrestre en Chine et en France. Nous avons utilisé l'unité \textit{km × kg} pour le transport maritime et terrestre, en estimant le poids de notre colis avec la somme des poids de l'ordinateur et de son emballage, soit 10,75kg.


\subsubsection{Création du processus}
Nous avons alors pu créer le processus \textit{Computer DELL T1700/3620} visible en figure~\ref{fig:computer-process} en intégrant l'ensemble des processus que nous venons de décrire. Il représente la fabrication d'un ordinateur de ce modèle. Nous complétons ce processus par un second processus \textit{Use of computer DELL T1700/3620 at ENSEIRB-MATMECA} pour prendre en compte à la fois les 600 ordinateurs mais aussi leur consommation électrique selon les données fournies dans le sujet~\cite{TP2_ACV_ENSEIRB-MATMECA}. Ainsi nous pouvons créer le \textit{product system}\footnote{Un product system est un ensemble de processus interconnectés qui représentent un système de produits ou de services. Il permet d'analyser les flux de matières et d'énergie tout au long du cycle de vie d'un produit.} associé et réaliser le calcul d'impact.

\subsection{Six cent Raspberry Pi couplés à six serveurs}
Nous considérons dans notre étude que 600 Raspberry Pi 4 Rev B sont couplés à 6 serveurs Dell Precision 7920T. Nous allons modéliser l'impact environnemental de cette solution en utilisant directement les processus qui nous ont été fournis. A partir de ces processus, nous avons pu en créer un autre, \textit{Use of Raspberry + Server at ENSEIRB-MATMECA}, correspondant à l'utilisation d'un système informatique composé de Raspberry Pi et de Serveurs 48 coeurs à l'ENSEIRB-MATMECA. Celui ci modélise donc à la fois la fabrication et le transport initial du matériel informatique ainsi que sa consommation électrique au court de sa durée de vie.\\
\textcolor{blue}{[Question 8]}~\textcolor{red}{On considère que 6 serveurs sont suffisants pour piloter 600 clients légers constitués de Raspberry Pi.\\ 8. Regarder et commenter le processus Raspberry Pi 4 Rev B. Faire de même avec le processus Serveur 48 cœurs. Remarque : Le processus mounting correspondant ici à l’assemblage des composants sur le circuit imprimé (placement des composants, brasure, nettoyage des cartes assemblées à l’issue du processus, etc.). Pour l’ordinateur et le serveur, il est déjà directement intégré au niveau des composants.}


\begin{figure}[H]%[htbp]
    \centering
    \includegraphics[width=\linewidth, trim=80 0 80 90, clip]{img/graph-dell-vs-our.png}
    \caption{Comparaison entre le bilan carbone de Dell~\cite{Dell2018} et celui de notre modélisation}
    \label{fig:computer-impact}
\end{figure}
\section{Résulats}
\subsection{Ordinateur Dell de type tour T1700/3620}
\label{sec:computer}

\textcolor{blue}{[Question 6-1 et 7-b]}~La figure~\ref{fig:computer-impact} présente la comparaison entre le bilan carbone de Dell et celui de notre modélisation. Nous pouvons observer que la carte représente près de 60\% du bilan carbone total d'après DELL, tandis que notre modélisation indique que la carte mère et les autres cartes représentent environ 70\% du bilan carbone total. Néanmoins ils sont équivalent en termes de CO\textsubscript{2}e émis comme présenté dans le tableau~\ref{tab:emissions}.\\
\begin{table}[H]
    \centering
    \begin{tabular}{|p{2.5cm}|p{2.4cm}|p{2.4cm}|}
    \hline
    \textbf{Composant} & \textbf{Émission d'après DELL} & \textbf{Émission d'après nos résultats} \\
    \hline
    Transportation & 28{,}89 & 5{,}421 \\ \hline
    Chassis & 28{,}89 & 28{,}075 \\ \hline
    Power Supply Unit & 70{,}62 & 39{,}076 \\ \hline
    Packaging & 3{,}852 & 1{,}997 \\ \hline
    Hard Drive & 10{,}272 & 16{,}149 \\ \hline
    Main Board \& Other Boards & 199{,}02 & 197{,}501 \\
    \hline
    \textbf{Total} & \textbf{341{,}544} & \textbf{288{,}219} \\
    \hline
    \end{tabular}
    \caption{Émissions de CO\textsubscript{2}e pour chaque composant de l'ordinateur Dell T1700/3620 selon DELL~\cite{Dell2018} et selon nos résultats}
    \label{tab:emissions}
\end{table}
Nous avons également observé que le transport représente une part négligable d'après nos résultats, alors qu'il représente près de 10\% du bilan carbone d'après DELL. \\
Au total, nous avons obtenu un bilan carbone de 288,219 kg CO\textsubscript{2}e pour un ordinateur Dell de type tour T1700/3620, tandis que DELL annonce un bilan carbone de 341,544 kg CO\textsubscript{2}e.


\begin{figure}[H]%[htbp]
    \centering
    \includegraphics[width=\linewidth]{img/graph-frabr-usage-dell-vs-us.png}
    \caption{Comparaison entre les émissions de CO\textsubscript{2}e pour la phase de fabrication et d'utilisation de l'ordinateur Dell T1700/3620 selon DELL et selon nos résultats}
    \label{fig:fab-impact}
\end{figure}

\textcolor{blue}{[Question 7-a-1]}~La figure \ref{fig:fab-impact} nous permet quant à elle de comparer les émissions de CO\textsubscript{2}e pour la phase de fabrication et d'utilisation de l'ordinateur Dell T1700/3620 selon DELL et selon nos résultats. Nous pouvons observer que la phase de fabrication représente environ 50\% du bilan carbone total d'après DELL, tandis que notre modélisation indique que la phase de fabrication représente environ 75\% du bilan carbone total.\\
\begin{table}[H]
    \centering
    \begin{tabular}{|p{2.5cm}|p{2.5cm}|p{2.5cm}|}
    \hline
    \textbf{Catégorie} & \textbf{Émission d'après DELL} & \textbf{Émission d'après nos résultats} \\
    \hline
    Fabrication & 341{,}544 kg eq CO\textsubscript{2} & 288{,}218 kg eq CO\textsubscript{2} \\ \hline
    Usage & 296{,}604 kg eq CO\textsubscript{2} & 101{,}027 kg eq CO\textsubscript{2} \\ \hline
    Consommation annuelle & 139{,}6 kWh & 342{,}7 kWh \\
    \hline
    \end{tabular}
    \caption{Comparatif des émissions et consommation entre la phase de fabrication et d'utilisation de l'ordinateur Dell T1700/3620 selon DELL et selon nos résultats}
    \label{tab:comparatif}
\end{table}
Nous pouvons également constater sur la table~\ref{tab:comparatif} l'émissions données par DELL est bien supérieur à la notre que ce soit pour la phase de fabrication ou d'utilisation. alors que leurs consommation annuelle annoncé est bien inférieure à la notre.

\begin{figure}[H]%[htbp]
    \centering
    \includegraphics[width=\linewidth]{img/graph-other.png}
    \caption{Impact de l'utilisation et de la fabrication des ordinateurs sur le réchauffement climatique, les radiation ionisantes, la consommation d'eau et la rareté des ressources minérales}
    \label{fig:other-impact}
\end{figure}
\textcolor{blue}{[Question 7-c-1]}~Afin de mieux comprendre l'impact environnemental de l'ordinateur Dell T1700/3620, nous avons également étudié d'autres facteurs d'impact tels que le réchauffement climatique, les radiations ionisantes, la consommation d'eau et la rareté des ressources minérales. La figure~\ref{fig:other-impact} présente les résultats de cette analyse. Nous pouvons observer que la phase d'utilisation a un impact significatif sur les radiations ionisantes et la consommation d'eau, tandis que la phase de fabrication a un impact plus important sur le réchauffement climatique et la rareté des ressources minérales.
    
\subsection{Six cent Raspberry Pi couplés à six serveurs}
\begin{figure}[H]%[htbp]
    \centering
    \includegraphics[width=\linewidth]{img/pi-server-usage-vs-fabrication.png}
    \caption{Impact de l'utilisation et de la fabrication de la méthode Raspberry Pi sur le réchauffement climatique, les radiation ionisantes, la consommation d'eau et la rareté des ressources minérales}
    \label{fig:pi-server-usage-vs-fabrication}
\end{figure}
En comparant cette fois-ci l'impacte de la phase de fabrication et d'utilisation des Raspberry Pi et des serveurs, nous pouvons observer figure~\ref{fig:pi-server-usage-vs-fabrication} que cette méthode présente exactement la même répartition que l'ordinateur Dell T1700/3620. 

\begin{figure}[H]%[htbp]
    \centering
    \includegraphics[width=\linewidth]{img/graph-pi-server.png}
    \caption{Emissions de CO\textsubscript{2}e pour fabrication des rasberry Pi, des serevurs et leurs utilisation}
    \label{fig:graph-pi-server}
\end{figure}
Le graphqiue \ref{fig:graph-pi-server} présente les émissions de CO\textsubscript{2}e pour la fabrication des Raspberry Pi, des serveurs et leurs utilisation. Nous pouvons observer que la phase de fabrication des serveurs est plus faible que celle des Raspberry Pi, avec des émissions presquent équivalente a l'utilisation de la solution, c'est dire d'environ 4800 kg CO\textsubscript{2}e. 

\subsection{Comparaison entre les 2 stratégies}

\begin{figure*}[htbp]%[htbp]
    \centering
    \includegraphics[width=\linewidth]{img/8vs6vsDELL.png}
    \caption{Comparaison de l'impacte environnemental de six cent Raspberry Pi couplés à six serveurs, à huit serveurs et de 600 ordinateurs Dell de type tour T1700/3620}
    \label{fig:8vs6vsDELL}
\end{figure*}

La figure \ref{fig:8vs6vsDELL} représente la différence de l'impact environnemental des 2 stratégies, avec différentes quantités de serveurs. \\
Nous pouvons observer que la stratégie consistant à installer 600 ordinateurs Dell de type tour T1700/3620 a un impact environnemental environ 10 fois supérieur aux autres scénarios étudiés, et ce quelle que soit la catégorie d'impact. \\
Parmi les 2 scénarios implémentant des Raspberry Pi avec des serveurs, si les impacts environnementaux du scénario optant pour 8 serveurs sont légèrement supérieurs, ils restent cependant très proches de celui n'en ayant que 6.

\section{Discussion}

\subsection{Ordinateur Dell de type tour T1700/3620}
\textcolor{blue}{[Question 6-2]}~Dans la section~\ref{sec:computer}, nous avons pu constater que les empreintes carbones que nous avons calculées et celle fournie par DELL sont du même ordre de grandeur, bien que celle de DELL soit légèrement supérieure. Cette différence, principalement visible dans le transport, pourrait provenir d'une différence dans la façon de modéliser l'acheminement des ordinateurs. En effet, DELL indique que les ordinateurs sont assemblés en Europe, tandis que nous avons considéré celui-ci en Chine, ce qui pourrait présenter des différences d'acheminement des composants par rapport à notre modélisation.\\ \\
\textcolor{blue}{[Question 7-a-2]}~Nous avons aussi pu observer une différence significative entre les émissions de CO\textsubscript{2}e pour la phase de fabrication et d'utilisation, avec pourtant une tendance inverse pour la consommation annuelle. Celle-ci peut s'expliquer par le fait que nos calculs dépendent d'une consommation électrique française, alors que DELL présente ses études au sein de l'UE. Or, en France, l'utilisation de centrales nucléaires pour la production d'électricité rend la consommation électrique bien plus faible en termes d'émissions de CO\textsubscript{2}e.
\textcolor{blue}{[Question 7-c-2]}~Les résultats de l'analyse de cycle de vie montrent que la phase d'utilisation a un impact significatif sur les radiations ionisantes et la consommation d'eau, alors qu'elle est bien plus faible sur le réchauffement climatique. Ceci s'explique par le mix énergétique français qui, par le nucléaire, est très décarboné mais produit en revanche des déchets nucléaires qui eux sont radioactifs et nécessitent beaucoup d'eau pour le refroidissement des centrales.
Les ressources minérales sont quant à elles impactées par la phase de fabrication, ce qui s'explique par la nécessité d'extraire des métaux et des minéraux rares pour la fabrication des composants électroniques.


\subsection{Six cents Raspberry Pi couplés à six serveurs}
\textcolor{blue}{[Question 9]}~Les résultats précédents nous ont permis de constater que la phase de fabrication et la phase d'usage des Raspberry Pi et des serveurs ont un impact environnemental similaire à celui de l'ordinateur Dell T1700/3620 sur l'ensemble des facteurs étudiés. Ce qui peut facilement s'expliquer par le fait qu'il s'agit de procédés et de matériaux similaires dans les deux cas. Nous observons néanmoins ici que le plus impactant dans cette solution est les Raspberry Pi, et ceci est dû à leur grande quantité.
\subsection{Comparaison entre les 2 stratégies}
\textcolor{blue}{[Question 10-1]}~Dans cette étude, nous avons modélisé l'impact environnemental pour six serveurs. Les résultats obtenus dans la section précédente montrent que cette solution est prometteuse avec un impact environnemental bien plus faible que celui de l'ordinateur Dell T1700/3620. Cependant, en cas de panne, cette quantité peut s'avérer insuffisante. \textcolor{blue}{[Question 11]}~Nous avons donc modélisé l'impact environnemental pour huit serveurs, ce qui a présenté une très légère augmentation de l'impact de la solution comme le présente la figure~\ref{fig:8vs6vsDELL}, ceci ne changeant donc pas notre conclusion précédente. En effet, nous avons pu observer que l'impact environnemental de la solution Raspberry Pi est bien plus faible que celui de l'ordinateur Dell T1700/3620.\\
Notons néanmoins que bien que la solution des Raspberry Pi semble bien meilleure, sa mise en place pose beaucoup de contraintes : le réseau complet de l'ENSEIRB-MATMECA doit être revu, l'ensemble de ses équipements doit être acheté ensemble, ce qui engendre un très grand investissement sans acompte. \\
\textcolor{blue}{[Question 10-2]}~Ainsi, bien que cette solution soit avantageuse sur le plan environnemental, sa mise en place est complexe et coûteuse.\\

<<<<<<< HEAD
\subsection{Impact carbone pour la scolarité d'un élève}
\textcolor{blue}{[Bonus 4]}~Après avoir calculé le bilan carbone de l'ensemble du parc informatique de l'ENSEIRB-MATMECA, nous pourrions aussi nous intéresser au bilan carbone lié à l'utilisation des services informatiques de l'établissement par un étudiant ou une étudiante pendant la durée de sa formation. Pour ce faire, nous allons vouloir ramener le bilan carbone de l'utilisation de l'infrastructure informatique de l'ENSEIRB-MATMECA par l'ensemble de l'établissement pendant la durée de vie du matériel à l'échelle d'une personne et de la durée de sa scolarité. Un certain nombre d'hypothèses vont devoir être faites : premièrement, nous supposerons que l'empreinte carbone se limite à l'ensemble des ordinateurs présents sur place et plus particulièrement à ce qui a été modélisé par \textit{Use of computer DELL T1700/3620 at ENSEIRB-MATMECA} (nous omettons par exemple les services en ligne) ; deuxièmement, nous supposerons que seuls les quelques 1200 élèves ingénieurs présents chaque année utilisent l'infrastructure informatique, qu'ils l'utilisent tous avec la même intensité (quelle que soit la filière donc), et que toutes les formations durent exactement 3 ans.
=======
\subsection{Impacte carbone pour la scolarité d'un élève}
\textcolor{blue}{[Bonus 4]}~Après avoir calculé le bilan carbone de l'ensemble du parc informatique de l'ENSEIRB-MATMECA, nous pourriont aussi nous intéresser au bilan carbone lié à l'utilisation des services informatiques de l'établissement par un étudiant ou une étudiante pendant la durée de sa formation. Pour se faire nous allons vouloir ramener le bilan carbone de l'utilisation de l'infrastructure informatique de l'ENSEIRB-MATMECA par l'ensemble de l'établissement pendant la durée de vie de matériel à l'echelle d'une personne et de la durée de sa scolarité. Un certain nombre d'hypothèses vont devoir être faites, premièrement nous supposerons que l'empreinte carbone se limite à l'ensemble des ordinateurs présents sur place et plus particulièrement à ce qui a été modélisé par \textit{Use of computer DELL T1700/3620 at ENSEIRB-MATMECA} (Nous omettons par exemple les services en ligne), deuxièmement nous supposerons que seuls les quelques 1200 élèves ingénieurs présents chaques années utilisent l'infrastructure informatique, qu'ils l'utilisent tous avec la même intensité (Quelque soit la filière donc), et que toutes les formations durent exactement 3 ans. \\
Ainsi quantifier l'utilisation des services informatiques de l'ENSEIRB-MATMECA en personnes$\cdot$années. Un élève utilisera donc 3 personnes$\cdot$années aun cours de sa scolarité, tandis que le système informatique qui sera utilisé par 1200 personnes pendant 5 ans offre $5\times1200=6000$ personnes$\cdot$années de quantité d'utilisation. Cet élève utilisera donc $\frac{3}{600}=0.5\%$ de la quantité d'utilisation de l'infrastructure au court de sa scolarité, ce qui engendrera un bilan carbonne de $389.245\times\frac{0.5}{100}\approx1.95$ kg eq CO$_{2}$.
>>>>>>> 83869c5 (Fin B4)

\subsection{Limites de l'étude}
\textcolor{blue}{[Question 10-3]}~L'ensemble des processus utilisés pour modéliser les ordinateurs est déjà intégré dans la base de données ecoinvent. Cependant, il est important de noter que ces processus peuvent ne pas être entièrement représentatifs des modèles spécifiques de l'ordinateur Dell T1700/3620. Par conséquent, il est possible que les résultats de l'analyse de cycle de vie ne reflètent pas fidèlement l'impact environnemental réel de ces ordinateurs. De plus, la modélisation des Raspberry Pi et des serveurs a été réalisée en utilisant des processus génériques, ce qui peut également introduire des incertitudes dans les résultats. Il est donc essentiel d'interpréter les résultats avec prudence et de considérer ces limitations lors de l'évaluation de l'impact environnemental du parc informatique de l'ENSEIRB-MATMECA.\\
\textcolor{blue}{[Bonus 6]}~En plus de cela, nous n'avons pas pris en compte les équipements externes (écrans, claviers, souris, etc.) qui peuvent également avoir un impact significatif sur l'empreinte carbone totale du parc informatique et pourraient nécessiter d'être changés selon leur compatibilité avec les Raspberry Pi. \\
Par exemple, un écran consomme entre 10 et 30 Wh~\cite{maitriserenergie2024}. En considérant que ceux-ci ne sont jamais éteints, cela représente environ 0,48 kWh par jour par écran, soit plus de 100 kWh par an pour l'ensemble du parc informatique. \\
Ainsi, pour rendre cette étude plus précise, il aurait fallu étudier leur compatibilité avec les Raspberry Pi et modéliser leurs impacts environnementaux.


% Conclusion
\section{Conclusion et ouvertures}
\textcolor{blue}{[Question 10-1]}~Dans notre étude, nous avons pu évaluer l'impact environnemental de deux solutions pour le parc informatique de l'ENSEIRB-MATMECA : 600 ordinateurs Dell de type tour T1700/3620 et 600 Raspberry Pi 4 Rev B connectés à 6 serveurs Dell Precision 7920T. Nous avons pu constater que le parc informatique actuel a un impact environnemental très élevé, avec un impact par étudiant durant sa scolarité qui s'élève à \textcolor{red}{XXXXX} kg CO\textsubscript{2}e. En revanche, la solution des Raspberry Pi couplés à des serveurs présente un impact environnemental bien plus faible, avec un impact divisé par 10.\\ \\
Néanmoins, nous avons pu observer que la mise en place de cette solution est complexe et coûteuse, nécessitant un investissement important. De plus, il est important de noter que l'impact environnemental des Raspberry Pi et des serveurs peut varier en fonction de leur utilisation et de leur configuration.\\
Ainsi, bien que la solution des Raspberry Pi soit prometteuse sur le plan environnemental, sa mise en place doit être soigneusement planifiée et évaluée pour garantir son efficacité et sa durabilité à long terme.\\ \\
De plus, il est important de noter que cette étude ne prend pas en compte l'impact environnemental des équipements périphériques tels que les écrans, les claviers et les souris. Ces équipements peuvent également avoir un impact significatif sur l'empreinte carbone totale du parc informatique et devraient être pris en compte dans une analyse plus complète de l'impact environnemental du parc informatique de l'ENSEIRB-MATMECA.\\
Il serait également intéressant d'étendre cette étude à d'autres établissements d'enseignement supérieur afin de mieux comprendre l'impact environnemental des parcs informatiques dans le secteur de l'éducation. Cela pourrait permettre d'identifier des opportunités d'amélioration et de sensibiliser davantage les établissements à l'importance de la durabilité dans le domaine numérique.\\
Enfin, il serait intéressant de réaliser une étude similaire sur d'autres types de matériel informatique, tels que les ordinateurs portables ou les serveurs, afin de mieux comprendre l'impact environnemental de ces équipements et d'identifier des solutions plus durables.

% % Références
\printbibliography
\addcontentsline{toc}{section}{Références}
% \bibliography{bibliography}

\appendix

\end{document}
